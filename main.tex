\documentclass[a4paper, 11pt]{article}
\usepackage{comment} % enables the use of multi-line comments (\ifx \fi) 

\usepackage{fullpage}
\usepackage[english]{babel}
\usepackage[utf8]{inputenc}
\usepackage{amsmath}
\usepackage{graphicx}
\usepackage[colorinlistoftodos]{todonotes}
\usepackage{array}
\usepackage{longtable}

\begin{document}
%Header-Make sure you update this information!!!!
\noindent
\large\textbf{Bryce Patel and Amanda Tran} \hfill \textbf{EE-393 Technical Writing} \\
\normalsize AA \hfill Professor Bade \\
4/27/2016 \hfill Julianne Peeling




\begin{center}{\Large\textbf{Simple English Wikipedia Offline App}}\end{center}
\section*{Problem Statement}
Many children do not have access to offline educational resources. While Wikipedia is useful for learning, it requires a connection to the internet and it can be overly technical and confusing for people without much background knowledge.

\section*{Solution}
This product will solve the problem of Wikipedia being confusing since it uses the lesser-known but still useful Simple English Wikipedia. Simple English Wikipedia is meant for people who are not very familiar with the English language and with the topics they’re reading about, which is perfect for learning African children. This product also solves the problem of requiring internet access since it would compress the Simple English Wikipedia data into an application that can be used offline.

\section*{Product Specifications}

\begin{center}
  \begin{tabular}{ | l | r |}
    \hline
    Cost & \$600 \\ \hline
    Return to Community & \$1 million \\ \hline
    Educational Benefits & 1.3 GPA increase \\
    \hline
  \end{tabular}
\end{center}
\begin{center}Table 1: Cost and benefit data\end{center}
\begin{center}\includegraphics{Wikipedia-logo-simple}\end{center}
\begin{center}Figure 1: Simple English Wikipedia logo\end{center}

\section*{Conclusion}
We should be rewarded grant money to fund the implementation of this app because , though it is cheap to develop and implement in the community, the gains are significant. The estimated monetary benefit to the community is 1,000,000 US Dollars per year. This money comes in the form of economic growth from the economy diversifying from all the things they learn from this app. Also the money gained from advertisement will go directly back into fueling the community. Developing this app will only cost a lump sum fee of 600 US dollars. Volunteers will work to spreading this app throughout the community and educating the locals on how to use it for free. Finally the end goal of educating the kids will be met, which is apparent in a 1.3 grade point average increase in science and math classes after a year of using the app.

\end{document}
